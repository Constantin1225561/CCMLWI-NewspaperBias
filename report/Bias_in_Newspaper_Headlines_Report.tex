%
%  This is an example LaTeX file. The percent sign is used to mark the
% start of a comment.
%
%  (modified by George Kachergis from Michael Weeks' example_final_report2.tex)
%
\documentclass[final]{ieee}
\usepackage{graphics}
\journal{Final Project Report}

\title[journalExample]{Bias in Newspaper Headlines}
\author[Lastname]{%
   Timo van Niedek\member{Student} \\
   Constantin Br\^{i}ncoveanu\member{Student}
    \authorinfo{%
     Cognitive Computational Modeling of Language and Web Interaction,\\
      SOW-MKI61-2016-SEM2-V, 13th July 2017, Dr. G.E. Kachergis. \\
      email: \mbox{c.brincoveanu@student.ru.nl},\mbox{timo.niedek@student.ru.nl}} 
}



\begin{document}



\maketitle


\begin{abstract}
TODO
%The abstract should be no more than 150 words.
%It is a short summary of the paper. If you had to
%re-state what your paper says in 150 words or less, what would you
%say? By the way, I recommend writing the abstract LAST, since it is
 %    easier this way.
%      For a conference paper, most people will read the abstract to see
 %     if they find it interesting enough to read the whole paper. This
  %    makes a lot of sense if you go to a conference in a topic that
  %    interests you, but find that there are 100+ other papers.
\end{abstract}

\section{Introduction}\label{sec:intro}

Although newspapers tend to claim neutrality, it is almost impossible to find a source that does not contain any stereotypes or biases in some form Biases in news headlines pose a danger in that they can propagate stereotypes to the general public. Detecting biases manually requires an enormous amount of effort to code news headlines and apply theoretical bias frameworks. It is infeasible to apply this process to the vast amounts of news stories that are released every day. 

An alternative approach that has gained interest recently is to detect biases automatically using machine learning techniques. These systems employ natural language processing techniques to detect one of the many ways in which bias can be present in texts. One such method is sentiment analysis, in which the polarity of the text is determined, either positive, negative or neutral. Sentiment analysis is typically applied to highly subjective texts for which the polarity is very clear and has high variance across difference texts. The difficulty with using sentiment analysis on news headlines to detect bias is that the polarity scores lie closely together, and therefore do not clearly indicate a bias on its own. News outlets report negative events such as war, death or protests more often than positive events, which makes it difficult to compare polarity scores. Additionally, bias detection using polarity as a measure is not enough since negative scores appear often, even unbiased reports. We therefore propose a technique that investigates the difference between the headline sentiment and the mean of the polarity scores for a specific topic. 

Another method of investigating news bias is to measure the amount of flavor words used by a specific news source. These flavor words are adjectives, adverbs, comparatives and superlatives. The motivation behind this method is that these words can carry bias, and a truly objective report would need less of them. Our method compares the amount of flavor in a headline with the mean of the flavor across one topic to identify which news sources use significantly more flavor as an indicator of bias.

Our third and final approach is to classify news headlines into their respective sources, which will indicate how distinct a news source is. An unbiased news outlet would be very difficult to classify, whereas a news outlet that consistently uses biased language can easily be classified.

%Our goal in this project is to apply machine learning techniques to find those biases. We specifically aimed at looking at minorities, such as refugees, and we also examined general politics and anything that might be controversial.

%We approached this task by firstly selecting newspaper headlines that belong to a certain topic. This was done by Topic Detection, which will be explained in detail later in this paper. After that selection, a Sentiment Analysis revealed the biases of the individual newspapers.

%Another perspective on biases was achieved by implementing a classifier, which revealed further differences between the respective newspapers.
%     The introduction should get the reader interested, explain your
 %    motivations, and provide a quick guide to the rest of the paper.
     
 %     Why is your topic important? Convince us!
 %     Where is it used? What are the applications.
  %    What will you talk about, and what did you do?

  %   Include an overview of the rest of your paper (e.g., section 2
     %       covers...section 3 presents...).

\subsection{Definition of Bias}
"Bias in cognitive science is generally defined as a deviation from a norm, deviation from some true or objective value."
%TODO: add source
We examined deviations from the norm by calculating the mean topic sentiments, as well as the mean topic flavours, and then looking at the differences of the respective domains from those mean values.

A second way to detect biases was domain classification.
%TODO: describe better
            
            
\section{Background}
 
TODO
% Include any relevant and specific info,
 %e.g. hardware statistics, equipment used.
% There is {\bf always} some background to cover.
%            
%           Describe what other people had to say on this topic(s)
%            (be sure to cite your references, and quote as appropriate).
%           Describe what other people did on this topic (or related topics).
%           In the references section, cite the papers/books that you used.
%Talk about problems and shortcomings of their work.
%Talk about how your work is different and better.
%
%Here are some citation examples. 
%This is a book about VLSI \cite{Weste93}.
%Also, the references contain a good conference paper \cite{LiY88},
%and a good journal article \cite{BiS92}.
%Anything you found useful.
%Include textbooks from class if you want.


\section{Project}

Our project consists of several approaches which let us detect biases and analyze newspapers from different perspectives. In this section, we are going to provide information about the dataset we used, as well as describe our approaches, ranging from general sentiment analysis, topic detection and sentiment and flavor analysis for the specific topics, to domain classification.

\subsection{Data}

We got data from Newspaper APIs. Firstly, we looked at the Kaggle News Aggregator Dataset. %TODO: add source
This dataset did not fulfill our requirements, because it only covered a quite small timespan in 2014 and it had a limited number of content categories, and no politic content.

Therefore, we started looking for alternative datasets, which led us to the discovery of the Reddit API. We selected newspaper headlines from "r/worldnews" ranging from 2016 to 2017. From those headlines, we further selected the ones that came from the most occurring domains. %TODO: explain in greater detail

\subsection{Design}

%Describe your approach to the problem.
%Describe what you did.
%            What you already had (and where it came from).
%            What you added/changed.
%            For parts, include close-up drawings (e.g. Magic screenshots).

\subsubsection{Data}
Firstly, we looked at the Kaggle News Aggregator Dataset\cite{Lichman13}. The advantage of this dataset is that it contains over 400k news stories from 
This dataset did not fulfill our requirements, because it only covered a quite small timespan in 2014 and it had a limited number of content categories, and no politic content.

Therefore, we started looking for alternative datasets, which led us to the discovery of the Reddit API. We selected newspaper headlines from "r/worldnews" ranging from 2016 to 2017. From those headlines, we further selected the ones that came from the most occurring domains. %TODO: explain in greater detail

\subsubsection{General Sentiment Analysis}

% Note: in this section, there shouldn't be any results or discussion, we will add that in the Results & Summary sections
% This section is only for describing the methods
One of our first results was a general Sentiment Analysis. %TODO: explain how it works and how we used it etc.
We aggregated the results by averaging the sentiments of the headlines for each domain. Those average values ranged from -0.3 to -0.1. %TODO: insert exact values, and maybe also add statistically significance tests.
This revealed some interesting differences between the domains. It became clear that some domains had very negative average sentiments, because they focused on negative topics such as war and general world news, whereas others have relatively positive average sentiments, because they focused on economics or science news. %TODO: add examples and talk more about it
%TODO: add results in Results section

\subsubsection{Topic Detection}

As a more fine-grained measure of bias, we consider the sentiment analysis per topic. To extract the topics from our data set, we use the biterm topic model (BTM) \cite{BTM13}, which is especially suitable for short texts. BTM uses the word co-occurrence patterns (biterms) to create a model over the entire corpus, thus solving the problem of sparse word co-occurrences within one document. The model parameters are estimated using Gibbs sampling. We create a vocabulary of the top 5000 most occurring words in the corpus, and run the Gibbs sampling algorithm to model $T = 20$ topics. For the Dirichlet priors, we use $\alpha = 50/T = 2.5$ as recommended by \cite{FST13} and $\beta = 0.01$ in order to make the topics contain a small number of highly distinct words. Due to limitations in computational power and the large size of the data set, we were limited to 250 iterations of Gibbs sampling, which took approximately nine hours to complete.

The resulting topics were validated using a small-scale study in which three participants were asked to define a topic from the top twenty most distinct keywords as found by the BTM, and give a rating on a scale from 1 to 7 for how coherent the topic was. We selected a subset of the twenty topics based on coherence and potential for controversy to perform further analysis over.

\subsection{Evaluation}
TODO
 %         What did/didn't work?
 %         Include graphs, equations, pictures, etc. as appropriate


\subsubsection{Topic Detection}
%TODO

\subsubsection{Sentiment Analysis}
%TODO

\subsubsection{Flavour}
%TODO

\subsubsection{Domain Classification}
We implemented a Bag of Words model with a Random Forest Classifier. For this task, we handpicked six domains. Those were the five most occurring domains: BBC, CNN, Independent, Reuters, and The Guardian. We additionally selected Foxnews as a sixth domain, hoping for it to be a contrast to the other five ones.
%TODO: talk more, and add Results

%{\bf Each team member must submit an individual report.}

%See the ``paper summary feedback'' on the class website for
%useful examples of what to do when writing a technical document.

%\hrule

%Remember your audience - your paper should be understandable by any CS
%student (at your level) who has taken this class.

%Things to include in the report:
%\begin{itemize}
%    \item Do not use bullet points unless absolutely necessary!
%    \item Pictures
%    \item Your observations and measurements
 %   \item  Equations
%    \item Graphs
 %   \item Figures
 %   \item Simulation, model 
%\end{itemize}

%Note: If you use color graphs, make sure you use a color printer when
%printing the report! I have seen several reports that say things like
%``the blue dots represent ..., while the red ones represent...'', only to
%have the figure printed in grey-scale.


%\hrule

%\subsection{ACM style}
%So here is my
%incomplete ``online first'' citation in ACM style:

%\verb"[Boisvert07]" Boisvert, R. F., Irwin, M. J., and Rushmeier, H. 2007.
%Online First: Evolving the ACM Journal Distribution Program.
%{\it Communications of the ACM}, Article unknown (September 2007), 2 pages.
%(Pages 19-20). DOI = unknown. http://doi.acm.org/unknown.

%------------------------------------------------------------------------

%See the following guide to documenting your sources <quote.html>. 

%------------------------------------------------------------------------
%Last update: May 2, 2008

%    For references, use the following style:

%Note: You can use [1] instead of [Lyons97], if you prefer.

\section{Results}
%            Include relevant observations, measurements, and statistics.
%            For example, for the VLSI Class: Include statistics such as
%            timing information if available by simulation, or if not,
 %           your own analysis about critical path, delays, and clock
%            cycles. Be sure to include size information: the total size
 %           of the circuit measured (X lambda by Y lambda), and the
 %           transistor count. 

\subsection{Sentiment and Flavor Analysis}
%TODO

\subsection{Domain Classification}
%TODO
            
\section{Summary}
TODO
%           Try to draw together the intro, background, and project
 %           sections.
 %          How do they all relate together? (They may appear to be
 %           disjoint sections to an unfamiliar reader).
 %          Restate important results.

\section{Conclusions}
There are measurable biases in newspaper headlines. Classification yielded surprisingly good results. The detected differences between the respective domains are statistically significant.

Sentiment Analysis is not accurate enough to detect the true sentiment value for each headline. In fact, the individual values often appear disappointingly inaccurate. However, if the number of headlines is big enough, the aggregated mean values per domain show significant differences from the general mean.

There are many more possible approaches to detecting newspaper biases, and we only scratched the surfaces of this broad topic. Further research is definitely going to reveal more. %TODO: talk more



%           You can combine Summary and Conclusions as you see fit.
%           What was accomplished and learned.
 %          What you would have done differently.
 %          Future work.
           
%The references must be in the same 2-column format as the rest of your
%paper.



% Now here is the reference section.
\begin{thebibliography}{99}
  \bibitem{BTM13} Xiaohui Yan, Jiafeng Guo, Yanyan Lan and Xueqi Cheng, 
   ``A Biterm Topic Model for Short Texts'', {\it Proceedings of the 22nd international conference on 
   World Wide Web}, ACM, 2013, pages 1445--1456.

  \bibitem{LDA03} D. M. Blei, A. Y. Ng and M. I. Jordan, ``Latent Dirichlet Allocation'' {\it Journal of machine Learning research},  Jan. (2003), pp.~993--1022.
   
   \bibitem{FST13} Thomas L. Griffiths and Mark Steyvers, 
   ``Finding scientific topics'', {\it Proceedings of the National academy of Sciences}, vol.~101, suppl 1, 2004 pages 5228--5235.


  % Book
  %\bibitem{ImABook} Joe Author, {\it Joe's Book About Stuff}, Publisher Press,
  %Atlanta, GA, 2008.
  
  %\bibitem{ImAConfPaper} Jane Scientist and Jake G. Student, 
   %``A Study of Conference Papers'', {\it Proceedings of the 3rd International
    %Conference on Research}, Paris, Texas, USA, January 4-6, 2008, pages 112-115.
    
    %\bibitem{ImAJournalPaper} Jane Scientist, Joe Author, and Jake G. Student,
    %``Journal Papers are Longer than Conference Papers'', 
    %{\it IEEE Transactions on Research}, Volume 4, Number 1, pages 212-235,
    %March 2008.
  
  
  %\bibitem{Weste93} Neil H. E. Weste and Kamran Eshraghian, {\it Principles
  %of CMOS VLSI Design}, 2nd ed. Reading, MA: Addison-Wesley, 1993.

  %Example of a Conference Paper
  %\bibitem{LiY88} R. A. Lincoln and K. Yao, ``Efficient Systolic Kalman
  %Filtering Design by Dependence Graph Mapping,'' in {\it VLSI Signal
  %Processing, III}, IEEE Press, R. W. Brodersen and H. S. Moscovitz Eds.,
  %1988, pp.~396--410.

  % Example of a Journal Paper
  %\bibitem{BiS92} C. H. Bischof and G. M. Shroff, ``On Updating Signal
  %Subspaces,'' {\it IEEE Trans. on Signal Processing}, vol.~40, no.~1,
   %Jan. 1992, pp.~96--105.

  %\bibitem{Lyons97} Richard Lyons, {Understanding Digital Signal Processing},
%Addison-Wesley, 1997.

  %\bibitem{Strang97} Strang and Nguyen, {\it Wavelets and Filter Banks}, Revised
%Edition, Wellesley-Cambridge Press, 1997.

%\bibitem{Weeks99} Michael Weeks, Beth Lumetta, Magdy Bayoumi, "The Black Jack
%Tutor Chip: Dealing From Idea to Silicon," /IEEE Potentials/, April/May
%1999, pages 38-42.

% example conference paper
%\bibitem{Zhang99} Guoqing Zhang, Mike Talley, Wael Badawy, Michael Weeks and
%Magdy Bayoumi, "A Low Power Prototype for a 3-D Discrete Wavelet
%Transform Processor," {\it IEEE International Symposium on Circuits and
%Systems (ISCAS '99)}, Orlando, Florida, May 30-June 2 1999, pages 80-83.

% example webpage
%\bibitem{Clarke04} Peter Clarke (Silicon Strategies), "Silterra demonstrates
%0.13-micron 8-Mbit SRAM", {\it EE Times},
%\texttt{http://www.eetimes.com/semi/news/}
%\texttt{showArticle.jhtml;}
%\texttt{jsessionid=}
%\texttt{0JJT0OEQDLM3MQSNDBGCKH0CJUMEKJVN?}
%\texttt{articleID=54201193},
%posted November 30, 2004 (5:54 AM EST), accessed November 30, 2004.

\bibitem{Lichman13} Lichman, M. (2013). {\it UCI Machine Learning Repository} \texttt{http://archive.ics.uci.edu/ml}. Irvine, CA: University of California, School of Information and Computer Science.

%You can use this PDF example
%\verb"http://mocha-java.uccs.edu/ieee/ieeeftp/"
%\verb"ieeecls.pdf", but follow these
%instructions. If you want to use LaTeX, there are directions at
%\verb"http://carmaux.cs.gsu.edu/"\verb"~mweeks/latex/" and an example file
%\verb"http://carmaux.cs.gsu.edu/"\verb"~mweeks/latex/"
%\verb"example_2col_jour.tex" that you can use as a template.


\end{thebibliography}

\newpage

% Make a blank page
\hbox{}


\newpage

\thispagestyle{empty}

\section*{Appendix}

\subsection*{Topics}

%You are allowed to have appendices, as needed. Appendices are mainly for
%code or mathematical derivations. You do not have to turn in all code
%used in your experiment; use your best judgement. You may want to
%include only relevant sections of code. Appendices do {\bf not} count in the
%page count. For example, if you have 4 pages of report, you may also
%turn in an appendix that is as long as you like. The appendix should be
%separate, with your name(s) on it. The appendix does not have to be in two-column
%format. 
%The appendix pages should be ordered, but do not have to be numbered.

\end{document}
