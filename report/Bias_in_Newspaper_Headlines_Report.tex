%
%  This is an example LaTeX file. The percent sign is used to mark the
% start of a comment.
%
%  (modified by George Kachergis from Michael Weeks' example_final_report2.tex)
%
\documentclass[final]{ieee}
\usepackage{graphics}
\journal{Final Project Report}

\title[journalExample]{Bias in Newspaper Headlines}
\author[Lastname]{%
   Timo van Niedek\member{Student} \\
   Constantin Br\^{i}ncoveanu\member{Student}
    \authorinfo{%
     Cognitive Computational Modeling of Language and Web Interaction,\\
      SOW-MKI61-2016-SEM2-V, 13th July 2017, Dr. G.E. Kachergis. \\
      email: \mbox{c.brincoveanu@student.ru.nl}} 
}



\begin{document}



\maketitle


\begin{abstract}
The abstract should be no more than 150 words.
It is a short summary of the paper. If you had to
re-state what your paper says in 150 words or less, what would you
say? By the way, I recommend writing the abstract LAST, since it is
      easier this way.

      For a conference paper, most people will read the abstract to see
      if they find it interesting enough to read the whole paper. This
      makes a lot of sense if you go to a conference in a topic that
      interests you, but find that there are 100+ other papers.
\end{abstract}

\section{Introduction}\label{sec:intro}
    
You can use this file to start your own LaTeX file,
and just delete the stuff you do not need. \LaTeX  is a lot like working
with HTML: you can specify where text effects begin, and where they end.

The project reports should be like conference papers: concise and
focussing on what you did.

Format: 
You can use the \verb"ieee.cls" file with \LaTeX. 
The ieee.cls format uses 11/16 inch margins (left and right),
3/16 inch between columns, 13/16 inch top margin, and 10/16 bottom
margin. A paper formatted with this may appear to be a bit shorter. Remember
that the paper is graded based on content, not how much paper is
used. 
{\bf This document is formatted with the \verb"IEEEconf.cls" file.}

You can also use the \verb"IEEEconf.cls" file. This file
format uses 1 inch margins for left, right, and top, and 
a $1 \frac{1}{8}$ inch bottom margin. It has $\frac{3}{8}$
inch between columns. 
%{\bf This document is formatted with the \verb"IEEEconf.cls" file.}

%If you are formatting your document yourself, 
%use 1 inch margins (left and right), 1 inch margins (top and
%bottom), 11 point times font for the main text, and use 10 point courier
%font for computer code. Use your judgement for other situations (for
%example indented, italics, and 10 point courier font for quotations).
%Single space your text. Make the text fully-justified (where the letters
%are aligned on both the left and right). The text should be in
%2-columns, with 3/8 of an inch of space between columns. Your paper should
%be 4 pages long. 
%Having only 3 pages is just fine, as long as the content is good.
%Anything 6 pages or longer will not be graded.


     The introduction should get the reader interested, explain your
     motivations, and provide a quick guide to the rest of the paper.
     
      Why is your topic important? Convince us!
      Where is it used? What are the applications.
      What will you talk about, and what did you do?

     Include an overview of the rest of your paper (e.g., section 2
            covers...section 3 presents...).
            
            
\section{Background}
 
 Include any relevant and specific info,
 e.g. hardware statistics, equipment used.
 There is {\bf always} some background to cover.
            
           Describe what other people had to say on this topic(s)
            (be sure to cite your references, and quote as appropriate).
           Describe what other people did on this topic (or related topics).
           In the references section, cite the papers/books that you used.
Talk about problems and shortcomings of their work.
Talk about how your work is different and better.

Here are some citation examples. 
This is a book about VLSI \cite{Weste93}.
Also, the references contain a good conference paper \cite{LiY88},
and a good journal article \cite{BiS92}.
Anything you found useful.
Include textbooks from class if you want.


\section{Project}

It is always a good idea to have text between a section header and
a subheading.

\subsection{Design}

Describe your approach to the problem.
Describe what you did.
            What you already had (and where it came from).
            What you added/changed.
            For parts, include close-up drawings (e.g. Magic screenshots).
\subsection{Evaluation}
          What did/didn't work?
          Include graphs, equations, pictures, etc. as appropriate


{\bf Each team member must submit an individual report.}

See the ``paper summary feedback'' on the class website for
useful examples of what to do when writing a technical document.

%\hrule

Remember your audience - your paper should be understandable by any CS
student (at your level) who has taken this class.

Things to include in the report:
\begin{itemize}
    \item Do not use bullet points unless absolutely necessary!
    \item Pictures
    \item Your observations and measurements
    \item  Equations
    \item Graphs
    \item Figures
    \item Simulation, model 
\end{itemize}

Note: If you use color graphs, make sure you use a color printer when
printing the report! I have seen several reports that say things like
``the blue dots represent ..., while the red ones represent...'', only to
have the figure printed in grey-scale.


%\hrule

\subsection{ACM style}
In the article "Online First: Evolving the ACM Journal Distribution
Program," the ACM Publications Board presents the idea that research has
moved to a primarily electronic realm, with print publications serving a
secondary role [Boisvert07]. This is a big shift in thinking; previously
the printed version was the standard, with on-line copies merely a
convenient access method. ACM citations will now include a Digital
Object Identifier (DOI), a permanent identification code (like an ISBN
number) which can be translated to a web address by services like
http://doi.acm.org.


Another significant change with the ACM style is to number the pages
starting at 1 for all articles. A paper with 10 pages will have page
numbers 1-10, regardless of where it appears in the journal. So an
article number must be introduced to properly reference the pages of the
journal. Also, the date refers to when the article is posted.

Ironically, while reading my print copy of CACM, I tried to find this
article on-line, only to realize that I do not have on-line access.
Perhaps I do have an account and password buried somewhere. Or maybe I
did not pay extra for it, since I do not read articles on-line (at least
without printing them first). The print version does not list an article
number, nor the DOI, and does not start at page 1. So here is my
incomplete ``online first'' citation in ACM style:

\verb"[Boisvert07]" Boisvert, R. F., Irwin, M. J., and Rushmeier, H. 2007.
Online First: Evolving the ACM Journal Distribution Program.
{\it Communications of the ACM}, Article unknown (September 2007), 2 pages.
(Pages 19-20). DOI = unknown. http://doi.acm.org/unknown.

%------------------------------------------------------------------------

%See the following guide to documenting your sources <quote.html>. 

%------------------------------------------------------------------------
%Last update: May 2, 2008

\subsection{Working with \LaTeX}
You can have text in {\it italics} font, or in {\bf bold} font,
\underline{underlined}, and even $\overline{overlined}$.
What if you want overlined text, without italics? This can
be done by using \verb"mathrm" in the 
$\overline{\mathrm{overlined}}$ specification.

Include tables and figures within your report, but be sure to 
refer to them within the text. Also, say where you got them
if you did not create them yourself. Table~\ref{tab:example_tab}
shows an example table, with baseball statistics for several
teams. We see that the Panthers are arguably best since they
have the most runs, the second most hits, and the fewest errors.

\begin{table}[!hbt]
\begin{center}
  \begin{tabular}{|r|c|c|c|}
     \hline
 & runs & hits & errors  \\
     \hline
Cardinals  & 2 & 2 & 1  \\
Panthers & 4 & 8 & 0  \\
Tigers  & 2 & 3 & 2  \\
Braves  & 3 & 10 & 3  \\
     \hline
  \end{tabular}
  \caption{An example table}
  \label{tab:example_tab}
\end{center}
\end{table}

What if you want to include a figure? 
Here is an example, figure~\ref{fig:phasor1}, that is saved in 
encapsulated postscript format. It shows an example of a vector
from the origin to point $(a,b)$.

\begin{figure}[!hbt]
  \centering
    %\scalebox{.9}{\includegraphics{phasor1.eps}}
  \caption{Here is an example vector.}
  \label{fig:phasor1}
\end{figure}


\subsection{Here is some Computer Code}\label{sec:code}

I like using the \verb"verbatim" specification for computer code. 
For example, here is something that appears in several
languages:

\begin{verbatim}
    for (int i=0; i< MAXVALUE-1; i++)
    {
        if (array[i] < array[i+1]) 
        {
            temp = array[i];
            array[i] = array[i+1];
            array[i+1] = temp;
        }
    }
\end{verbatim}

See how it makes the code stand out? I think it makes it
much easier to read, too. You should also explain why you
chose to include the code. That is, what are you trying to 
show by including it?


\subsection{Here is some Math}\label{sec:math}
This is different from the previous section, section~\ref{sec:intro}.
This section gives some examples of Math.

Using superscript:  2$^{n}$

Using subscript: x$_{0}$

If you use a character, but LaTeX complains about it, try putting a 
back-slash before it. For example, 
f = x\^{}y  uses the carat character. 
If you want to end a line, use 2 back-slashes.
If you want the backslash character $\backslash$ in your document,
this can be done, too.

Here's an equation:

\[ M^\bot = \{ f \in V' : f(m) = 0 \mbox{ for all } m \in M \}.\]

Here is how \verb"d2u/dx2" can be put in mathematics mode.
Use the dollar sign before and after math expressions when they
are in text $\frac{d^2 u}{dx^2}$, or use backslashes and square brackets
for a line of their own:

\[ \frac{d^2 u}{dx^2} \]

Here's another equation:

\[ \lim_{x \to 0} \frac{3x^2 +7x^3}{x^2 +5x^4} = 3.\]

Here's a summation:
\[ \sum_{k=1}^n k^2 = \frac{1}{2} n (n+1).\] 

and an integral:
\[ \int_a^b f(x)\,dx.\]

Here are some Greek letters:
$ \Delta \Psi \Phi $
and some lower case ones:
$ \delta \psi \phi \omega \pi \sigma \mu $.

%For more info, see
%http://www.maths.tcd.ie/\~{}dwilkins/LaTeXPrimer/





%    For references, use the following style:

%Note: You can use [1] instead of [Lyons97], if you prefer.

\section{Results}
            Include relevant observations, measurements, and statistics.
            For example, for the VLSI Class: Include statistics such as
            timing information if available by simulation, or if not,
            your own analysis about critical path, delays, and clock
            cycles. Be sure to include size information: the total size
            of the circuit measured (X lambda by Y lambda), and the
            transistor count. 
            
\section{Summary}
           Try to draw together the intro, background, and project
            sections.
           How do they all relate together? (They may appear to be
            disjoint sections to an unfamiliar reader).
           Restate important results.
\section{Conclusions}
           You can combine Summary and Conclusions as you see fit.
           What was accomplished and learned.
           What you would have done differently.
           Future work.
           
The references must be in the same 2-column format as the rest of your
paper.



% Now here is the reference section.
\begin{thebibliography}{99}

  % Book
  \bibitem{ImABook} Joe Author, {\it Joe's Book About Stuff}, Publisher Press,
  Atlanta, GA, 2008.
  
  \bibitem{ImAConfPaper} Jane Scientist and Jake G. Student, 
   ``A Study of Conference Papers'', {\it Proceedings of the 3rd International
    Conference on Research}, Paris, Texas, USA, January 4-6, 2008, pages 112-115.
    
    \bibitem{ImAJournalPaper} Jane Scientist, Joe Author, and Jake G. Student,
    ``Journal Papers are Longer than Conference Papers'', 
    {\it IEEE Transactions on Research}, Volume 4, Number 1, pages 212-235,
    March 2008.
  
  
  \bibitem{Weste93} Neil H. E. Weste and Kamran Eshraghian, {\it Principles
  of CMOS VLSI Design}, 2nd ed. Reading, MA: Addison-Wesley, 1993.

  %Example of a Conference Paper
  \bibitem{LiY88} R. A. Lincoln and K. Yao, ``Efficient Systolic Kalman
  Filtering Design by Dependence Graph Mapping,'' in {\it VLSI Signal
  Processing, III}, IEEE Press, R. W. Brodersen and H. S. Moscovitz Eds.,
  1988, pp.~396--410.

  % Example of a Journal Paper
  \bibitem{BiS92} C. H. Bischof and G. M. Shroff, ``On Updating Signal
  Subspaces,'' {\it IEEE Trans. on Signal Processing}, vol.~40, no.~1,
   Jan. 1992, pp.~96--105.

  \bibitem{Lyons97} Richard Lyons, {Understanding Digital Signal Processing},
Addison-Wesley, 1997.

  \bibitem{Strang97} Strang and Nguyen, {\it Wavelets and Filter Banks}, Revised
Edition, Wellesley-Cambridge Press, 1997.

\bibitem{Weeks99} Michael Weeks, Beth Lumetta, Magdy Bayoumi, "The Black Jack
Tutor Chip: Dealing From Idea to Silicon," /IEEE Potentials/, April/May
1999, pages 38-42.

% example conference paper
\bibitem{Zhang99} Guoqing Zhang, Mike Talley, Wael Badawy, Michael Weeks and
Magdy Bayoumi, "A Low Power Prototype for a 3-D Discrete Wavelet
Transform Processor," {\it IEEE International Symposium on Circuits and
Systems (ISCAS '99)}, Orlando, Florida, May 30-June 2 1999, pages 80-83.

% example webpage
\bibitem{Clarke04} Peter Clarke (Silicon Strategies), "Silterra demonstrates
0.13-micron 8-Mbit SRAM", {\it EE Times},
\texttt{http://www.eetimes.com/semi/news/}
\texttt{showArticle.jhtml;}
\texttt{jsessionid=}
\texttt{0JJT0OEQDLM3MQSNDBGCKH0CJUMEKJVN?}
\texttt{articleID=54201193},
posted November 30, 2004 (5:54 AM EST), accessed November 30, 2004.

%You can use this PDF example
%\verb"http://mocha-java.uccs.edu/ieee/ieeeftp/"
%\verb"ieeecls.pdf", but follow these
%instructions. If you want to use LaTeX, there are directions at
%\verb"http://carmaux.cs.gsu.edu/"\verb"~mweeks/latex/" and an example file
%\verb"http://carmaux.cs.gsu.edu/"\verb"~mweeks/latex/"
%\verb"example_2col_jour.tex" that you can use as a template.


\end{thebibliography}

\newpage

% Make a blank page
\hbox{}


\newpage

\thispagestyle{empty}

\section*{Appendix}
You are allowed to have appendices, as needed. Appendices are mainly for
code or mathematical derivations. You do not have to turn in all code
used in your experiment; use your best judgement. You may want to
include only relevant sections of code. Appendices do {\bf not} count in the
page count. For example, if you have 4 pages of report, you may also
turn in an appendix that is as long as you like. The appendix should be
separate, with your name(s) on it. The appendix does not have to be in two-column
format. 
%The appendix pages should be ordered, but do not have to be numbered.

\end{document}
